\documentclass[12pt]{article}
\usepackage{graphicx}
\graphicspath{{imgs/}}
\usepackage{xeCJK}
\setCJKmainfont{Noto Serif CJK TC}
\usepackage[top=2cm, bottom=2cm, left=2cm, right=2cm, a4paper]{geometry}
\usepackage{setspace}
\setlength{\parskip}{2pt}
\usepackage{moresize}
\usepackage{placeins}
\usepackage{indentfirst}
\usepackage{amsmath}
\usepackage{caption, subcaption}

\begin{document}

\begin{center}
    \huge \textbf{EDA Final Project Report}
    
    \vspace{10pt}
    
    \large \textbf{110511010 楊育陞 110511067 葉哲伍}
\end{center}

\section{ICCAD 繳交截圖}

\section{Introduction}

\indent We take the \textbf{problem B} of the ICCAD contest as our final project. The problem is "Power and Timing Optimization Using Multibit Flip-Flop". Our method to solve this problem is mainly based on \textbf{clustering} and \textbf{gain-based greedy algorithm}. The clustering algorithm we use is inspired by the "Effective Mean Shift Algorithm" described in the paper "Graceful Register Clustering by Effective Mean Shift Algorithm for Power and Timing Balancing" \cite{jiang}.

\section{Method}

\subsection{Flowchart}

\begin{figure}[htbp]
    \centering
    \includegraphics[width=0.9\textwidth]{flowchart2.png}
    \caption{Flowchart of our method}
    \label{fig:flowchart}
\end{figure}
\FloatBarrier

\subsection{Division of Labor}
\begin{itemize}
    \item 楊育陞: Debanking, Greedy banking, Finetuning based on utilzation
    \item 葉哲伍: Force-directed placement, Clustering, Placement legalization
\end{itemize}

\subsection{Detailed Explanation}

Our flow of the method is shown in Figure \ref{fig:flowchart}. We will explain each step in detail in the following subsections.

\subsubsection{Debanking all flip-flops}
In the initial given circuit, there are flip-flops of different bit-widths and different types. We first debank all flip-flops to single-bit flip-flops. The type of the single-bit flip-flop, called \textbf{base flip-flop} in our method, is chosen to be the single-bit flip-flop with the least cost calculated by the given cost function. The timing part of the cost is calculated by the q pin delay of the flip-flop in this step.

The debanking process is shown in Figure \ref{fig:debanking}. The debanked flip-flops are placed in the same position as the original flip-flops, and the connections are also kept the same. In this step, we allow the cells to overlap with each other.

\begin{figure}[htbp]
    \centering
    \includegraphics[width=0.7\textwidth]{debank.png}
    \caption{Debanking process}
    \label{fig:debanking}
\end{figure}

\subsubsection{Force-directed placement}
After debanking the flip-flops, we perform force-directed placement to determine the placement of the single-bit flip-flops. The force(weight) of each net is determined by the slack of the net. The first debanking step and this step are used to optimize the timing condition of the circuit.

In this problem, the cells are split into two types, combinational cells and flip-flops. The combinational cells are not allowed to move. So we should deal with two conditions when moving a flip-flop:

\begin{itemize}
    \item \textbf{All the connected cells of this flip-flop are combinational cells:} In this case, we can directly move the flip-flop to the timing optimal position.
    \item \textbf{There are flip-flops in the connected cells:} In this case, directly moving the flip-flop may not achieve the best timing condition since other connected flip-flops may move as well. So we should consider the movement of the connected flip-flops and the movement of the flip-flop itself. We use a iterative greedy method to solve this problem: We move one flip-flop at a time and fix the position of the other flip-flops. We repeat this process until the positions of all flip-flops are fixed or the distances converge.
\end{itemize}

For this step, we introduce a \textbf{"lock" mechanism} to achieve high efficiency. The lock mechanism is used to lock the position of the flip-flops that have been placed in the optimal position or the distances for several iterations have converged. The locked flip-flops will not be moved in the following iterations. Since the condition 1 discussed above will lock the flip-flop in the first iteration, \textbf{there is no need to consider which condition the flip-flop belongs to}, highly improving the efficiency of the algorithm.

\subsubsection{Clustering}

\subsubsection{Greedy banking in each cluster}

\subsubsection{Placement legalization}

\subsubsection{Finetuning based on utilization}

\subsubsection{Back to clustering with different parameters}

\section{Result}

\section{Conclusion}

\begin{thebibliography}{9}
    \bibitem{jiang} 
    Ya-Chu Chang, Tung-Wei Lin, Iris Hui-Ru Jiang, and GiJoon Nam. "Graceful Register Clustering by Effective Mean Shift Algorithm for Power and Timing Balancing." Proceedings of the 2019 International Symposium on Physical Design (ISPD '19), 2019.
\end{thebibliography}

\end{document}
